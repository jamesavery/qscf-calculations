\documentclass[a4paper]{article}
\usepackage{amsmath,amsfonts,amssymb,graphicx,a4wide}
\title{Qsim/FEM: Sanity check of dielectrics and conductors}
\author{James Avery (avery@diku.dk)}
\begin{document}
\begin{abstract}
  The over all purpose of this exercise is to build confidence that
  the calculations done on the OPV5-molecule in the single electron
  transistor environment (SET) are correct.  Specifically, it should
  be verified that the calculation including dielectric materials and
  electrodes with external potentials work as they should, and that
  the calculation of total energy taking external potentials into
  account also works.

  The OPV5-molecule in the SET environment is much too complicated to
  solve exactly, and it is not easily calculated using standard QM
  software.  However, in this case it is useful to use a single
  hydrogen atom exposed to classes of these environments as a
  check. When the $H^{q+}$ atom is situated at a distance greater than the
  orbital radius (approx. 6 Bohrs) from any interface, it can be well
  approximated by a point charge $q$, and we can compare the result
  obtained from Qsim/FEM with the analytic solution.
\end{abstract}

\section{Experiments}
\label{sec:experiments}

\subsection{$H^{q+}$ in vacuum}
\label{sec:test1}
This tests the sanity of the plain DFT/FEM code. 

\newcommand{\Evac}[1]{\ensuremath{E^{\text{vc}}\left(#1\right)}}
\newcommand{\Vext}[1]{\ensuremath{V^{\text{ext}}\left(#1\right)}}

\paragraph{Set-up:}\ \\
\begin{tabular}[t]{p{9cm}c}
Gas form is approximated by solving in a huge
box with Dirichlet 0 boundary conditions. The charge $q$ is varied (with many points) in the half-open interval
$[-2;1[$. & 
  \begin{tabular}{c}
    \includegraphics[width=5cm]{Hq-vacuum2d}    
  \end{tabular} 
\end{tabular}




\paragraph{Expected results:}
The total energy should approximately be a quadratic polynomial in the added charge $q$.
\begin{equation}
  \Evac{q} = \Evac{0} + aq + bq^2
\end{equation}

\paragraph{Results:}

\subsection{A point charge interacting with a single conductor}
\label{sec:test2}
This tests the sanity of the way I handle boundary conditions. 
\paragraph{Set-up:}\ \\
\begin{tabular}[t]{p{9cm}c}
Gas form is approximated by solving in a huge
box with Dirichlet 0 boundary conditions. 
The distance to the $(x,y) = (0,0)$ plane is varied (with many points) in the interval $[6;3]$, and the 
charge $q$ is varied (with only a handful points) in the half-open interval
$[-2;1[$. & 
  \begin{tabular}{c}
    \includegraphics[width=5cm]{Hq-conduct1}    
  \end{tabular} 
\end{tabular}

\paragraph{Expected results:}
\begin{equation}
  E(\Delta x,q) \simeq \Evac{q} + \frac{q}{2\Delta x}
\end{equation}

\subsection{A point charge interacting with a single dielectric}
\label{sec:test3}

This tests the sanity of the way I handle dielectrics. 

\paragraph{Set-up:}\ \\
\begin{tabular}[t]{p{9cm}c}
  The space is divided into two volumes, one with dielectric constant 1 (vacuum), in which the charge resides,
  and the second with dielectric constant $\epsilon$.

  The permitivity $\epsilon$ of the dielectric volume is varied in $[1;10000[$ (with few points).
  The distance to the $(x,y) = (0,0)$ plane is varied (with few points) in the interval $[6;3]$, and the 
  charge $q$ is varied (with few points) in the half-open interval
  $[-2;1[$. & 
  \begin{tabular}{c}
    \includegraphics[width=5cm]{Hq-dielect1}    
  \end{tabular} 
\end{tabular}

\paragraph{Expected results:}
\begin{equation}
  E(\Delta x,q,\epsilon) \simeq \Evac{q} + \alpha(\epsilon)\frac{q}{2\Delta x}
\end{equation}
where I believe $\alpha$ is
\begin{equation}
  \alpha(\epsilon) = \frac{\epsilon-1}{\epsilon+1}
\end{equation}
Why? Because I remember something along these lines, and we must have
\[
\alpha(1) = 0 \text{ and } \alpha(\infty) = 1
\]

For $\epsilon = 1$, we should get the same result as for vacuum in
Expt. \ref{sec:test1}, and for $\epsilon = 10000$, we should get the
almost the same result as for Expt. \ref{sec:test2}.



\subsection{A point charge between two conductors}
\label{sec:test4}

\paragraph{Set-up:}\ \\
\begin{figure}
  \centering
  \includegraphics[width=7cm]{Hq-conduct2}
  \includegraphics[width=7cm]{Hq-conduct2b}  
  \caption{
    {\bf Left}: The image charges induced by the two conductors and the point charge. 
    {\bf Right}: The external potential $V^{\text{ext}}$ as a function of $x$.
  }
\end{figure}



\paragraph{Expected results:}
\begin{equation}
  \label{eq:mirrormirror}
  \begin{split}
  E(\Delta x,V,q) &\simeq \Evac{q} + \sum_{k=1}^\infty \frac{(-1)^k}{2k\Delta x} + \Vext{\mathbf{X}_H}    \\
                  &= \Evac{q} -\frac{\ln 2}{2\Delta x} + \Vext{\mathbf{X}_H} \\
                  &= \Evac{q} -\frac{\ln 2}{2\Delta x} + \frac{V_s+V_d}{2}
  \end{split}
\end{equation}

\end{document}

% 
% Skrael kartofler - skiver - kog i 10 minutters tid - tag op, dryp af - floede, muskat - 2 knivspidser
% kartofler i, floede saa det lige daekker, salt og peber paa, bedst med hvid peber.
% Daek til med staniol, eller koger al piskefloeden ind - hvis der skal mere vaeske, saa brug vand.
% 200 grader, mindst en time.